%%
 % Packages 
 %%

% Styling TOC
\usepackage{tocloft}   			
\usepackage{mathtools}

% Mathe Packages
\usepackage{amsmath,amssymb,epsfig,amsthm,amsfonts,bbm,textcomp}
\usepackage{newclude}
\newcommand*{\bfrac}[2]{\genfrac{\lbrace}{\rbrace}{0pt}{}{#1}{#2}}
\newcommand{\bs}[1]{\boldsymbol{#1}}
\newcommand*{\MyPath}{../../bld}%

% Grafiken
\usepackage{graphicx,wrapfig}
\usepackage[font=small,aboveskip=0pt,belowskip=0pt,labelfont=bf]{caption}
%\captionsetup[table]{labelsep=period}

% Sprache Windows / deutsch
\usepackage[ansinew]{inputenc}
\usepackage{babel}				

% Layout
\usepackage{a4}
\usepackage{setspace}
\usepackage{lscape} % Landscape
\usepackage{color} %
%\definecolor{dunkelblau}{RGB}{16,36,78}
%\usepackage{ragged2e}

% Fussnoten
\usepackage[flushmargin,hang]{footmisc}
\usepackage{listings} 			% Code und so
%\usepackage[breaklinks]{hyperref}				% aktivieren fuer letzten Durchgang - gibt zwar eine Warnung, doch sind Zeilen schoen gebrochen
\usepackage{hyperref}
\usepackage{url}
%\usepackage[onehalfspacing]{setspace}

% PDF-Informationen
\hypersetup{
    colorlinks=true,
    citecolor=black,
    urlcolor=blue,
    linkcolor=black,
    linktoc=page,
    pdftitle={Empirical Topics in International Economics},
    pdfsubject={US Interwar Period Analysis using the Blanchard-Quah decomposition},
    pdfkeywords={US Interwar Period}
    pdfauthor={Michael Kilchenmann}
}

% Tabellen
\usepackage{longtable} 			% Lange, mehrseitige Tabelle
\usepackage{tabularx}    		% Package used to make variable width-columns, i.e.,
                    				% column widths are changed to fit the maximum width
                            % and text is wrapped nicely.
\usepackage{array,hhline}
\usepackage{booktabs}				% Linien zwischen Zeilen 
\usepackage{multirow}				% Verkn�pfe Zellen �ber Zeilen hinweg
\usepackage{dcolumn}				% Erlaubt variable Orientierung am Decimalpunkt innerhalb einer Zelle
\newcolumntype{.}{D{.}{.}{1}}

% References Teil
\usepackage{natbib}
\usepackage{ifthen} 				% Fuer Probekompilation und Zeitersparnis

% Umgebungsvariablen
\setcounter{secnumdepth}{5} % macht, dass alle Ueberschriften nummeriert werden

% Command Abk�rzungen
\newcommand{\be}{\begin{equation}}
\newcommand{\ee}{\end{equation}}
\newcommand{\ben}{\begin*{equation}}
\newcommand{\een}{\end*{equation}}
\newcommand{\E}{\mathbb E}
\newcommand*{\TitleFont}{%
      \usefont{\encodingdefault}{\rmdefault}{b}{n}%
      \fontsize{13}{13}%
      \selectfont}

\setlength{\abovecaptionskip}{0.1cm}
\setlength{\belowcaptionskip}{-2mm}

% Seiten Layout
\usepackage{float} 					% Um Ver�nderungen am Textumbruch um Abbildungen zu verhindern
\usepackage[top=2cm,bottom=2cm,left=3.5cm,right=3.5cm,a4paper]{geometry}  
														% Seitenaufbau

% Numbering
\renewcommand{\thesection}{\arabic{section}} 
\renewcommand{\thesubsection}{\arabic{section}.\arabic{subsection}} 
\renewcommand{\thetable}{\Roman{table}} 
\let\stdsection\section
%\renewcommand\section{\newpage\stdsection}


\makeatletter
% Punkt nach Sektion, aber nicht in Zitaten 
%\renewcommand{\@seccntformat}[1]{{\csname the#1\endcsname}.\hspace{0.5em}}
\long\def\@makefigcaption#1#2{%
 \vskip\abovecaptionskip
 \sbox\@tempboxa{\textbf{#1.} #2}%
 \ifdim \wd\@tempboxa >\hsize
 \textbf{#1.} #2\par
 \else
  \global \@minipagefalse
  \hb@xt@\hsize{\hfil\box\@tempboxa\hfil}%
 \fi
 \vskip\belowcaptionskip} 

% Float fuer Figures
\renewcommand{\figure}{\let\@makecaption\@makefigcaption\@float{figure}}

% Lange Beschreibung fuer Figures
\long\def\@maketblcaption#1#2{%
 \vskip\abovecaptionskip
 \begin{center}\normalsize\textbf{#1. #2}\end{center} %\begin{center}\small\bf#1 \\\normalsize#2\end{center}
 
 \vskip\belowcaptionskip} 

% Float fuer Tables
\renewcommand{\table}{\let\@makecaption\@maketblcaption\@float{table}}
\makeatother



% Footnotes
\makeatletter
\newlength{\myFootnoteWidth}
\newlength{\myFootnoteLabel}
\setlength{\myFootnoteLabel}{1.2em} % <-- can be changed to any valid value
\renewcommand{\@makefntext}[1]{%
 \setlength{\myFootnoteWidth}{\columnwidth}%
 \addtolength{\myFootnoteWidth}{-\myFootnoteLabel}%
 \noindent\makebox[\myFootnoteLabel][r]{\@makefnmark\ }%
 \parbox[t]{\myFootnoteWidth}{#1}%
}
\makeatother

% Appendix
\makeatletter
\newcommand\appendix@section[1]{%
  \refstepcounter{section}%
  \orig@section*{Appendix \@Alph\c@section: #1}%
  \addcontentsline{toc}{section}{Appendix \@Alph\c@section: #1}%
}
\let\orig@section\section
\g@addto@macro\appendix{\let\section\appendix@section}
\makeatother

% Abbildung Platz fuellen
\makeatletter
\newcommand\wrapfill{\par
\ifx\parshape\WF@fudgeparshape
\nobreak
\vskip-\baselineskip
\vskip\c@WF@wrappedlines\baselineskip
\allowbreak
\WFclear
\fi
}
\makeatother

% Theoreme
\newtheorem{thm}{Theorem}
\newtheorem{cor}[thm]{Corollary}
\newtheorem{lem}[thm]{Lemma}
\newtheorem{prop}[thm]{Proposition}
\newtheorem{defn}[thm]{Definition}
\newtheorem{rem}[thm]{Remark}
\newtheorem{hyp} {Hypothesis}


% Testlauf schneller kompilieren...
\newboolean{writingprog} 		%Deklaration
\setboolean{writingprog}{false} %Zuweisung
